\documentclass[a4paper,12pt]{article}
\usepackage[english]{babel}
\usepackage{amsmath}
\setlength{\parindent}{0pt}

\begin{document}
\section{In brief}
Objective is to redact notes on Groseclose and Snyder (1991) and provide comparative statics on the key hypotheses of the model. Summarize how these partial derivatives translate into observable implications for reduced form findings.

\section{Set-up}

\subsection*{Players:}
\begin{enumerate}
    \item Legislator: \textit{i}.
    \item Vote buyers: $p \in \{A, B\}$
\end{enumerate}

\subsection*{Outcomes:}
\begin{enumerate}
    \item $y = \{s, x\}$
\end{enumerate}

\subsection*{Preferences:}
For legislator $i$, denote the intensity of $i$'s preference of voting for $x$ over $s$ as:

$$v(i) = u_i(x) - u_i(s)$$

For vote buyer \{\textit{A}, \textit{B}\}, simply denote preferences over outcomes as:

$$W_{p \in \{A, B\}} = U_p(x) - U_p(s)$$

\subsection*{Strategies:}
Let $a(\cdot)$ and $b(\cdot)$ denote the bribe offer functions.

\end{document}