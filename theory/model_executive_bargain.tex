\documentclass[12pt,a4paper]{article}
\usepackage{setspace} \onehalfspacing
\usepackage[top = 1.25in, bottom = 1.25in, left = 1.25in, right = 1.25in]{geometry}
\usepackage[utf8]{inputenc}
\usepackage{amsmath}
\usepackage{amsfonts}
\usepackage{amssymb}
\usepackage{amsthm}
\usepackage{graphicx}
\usepackage{natbib}
\usepackage{caption}
\usepackage{subcaption}
\usepackage{float}
\usepackage{pdflscape}
\usepackage{booktabs}
\usepackage{dcolumn}
\usepackage{pdflscape}
\usepackage[hyphens]{url}
\usepackage{enumitem}
\usepackage[table]{xcolor}
\usepackage{authblk}
\usepackage{appendix}
\usepackage{titletoc}

\newtheorem{theorem}{Theorem}[section]
\newtheorem{lemma}[theorem]{Lemma}
\newtheorem{proposition}{Proposition}
\newtheorem{corollary}{Corollary}
\renewcommand{\theproposition}{\arabic{proposition}}
\newcommand{\real}{\mathbb{R}_+^n}

\begin{document}

The following model is adapted, with little to no modifications, from Banks (2000). We can model the executive bargain with the legislature as follows. A mayor wants to pass her proposal $x$. We can include under this category a set of policy proposals: a new legislation, an infrastructure project or budgetary approval. In order to implement her proposal, she must pass it in the city council through a simple majority rule. However, the mayor does not govern alone. An opposition wants to block the mayor's proposal and maintain the status quo, denoted as $y$. 

Each city councilor is characterized by a proposal valuation parameter $v_i$ for all $i \in N$, where $v_i > 0$ means that the mayor's proposal $x$ is preferred. Payoff gets realized when city councilor $i$ votes, independent of the outcome. For each councilor, the mayor and opposition set a bribe schedule

\begin{align*}
    a \in (a_1, ..., a_n) \in \real\\
    b \in (b_1, ..., b_n) \in \real+
\end{align*}

Solving through backward induction, given a bribe schedule $(a, b)$, councilor $i$ prefers to vote for the mayor's proposal ($x$) if $a_i + v_i > b_i$ and to maintain the status quo ($y$) otherwise. Since indifferent councilors vote for the status quo, party $b$ needs to only match bribes from $A$, adjusting for individual preferences, i.e. $b_i = a_i + v_i$. Therefore, the opposition solves



\end{document}