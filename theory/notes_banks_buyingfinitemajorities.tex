\documentclass[12pt,a4paper]{article}
\usepackage{setspace} \onehalfspacing
\usepackage[top = 1in, bottom = 1in, left = 1in, right = 1in]{geometry}
\usepackage[utf8]{inputenc}
\usepackage{amsmath}
\usepackage{amsfonts}
\usepackage{amssymb}
\usepackage{amsthm}
\usepackage{graphicx}
\usepackage{natbib}
\usepackage{caption}
\usepackage{subcaption}
\usepackage{float}
\usepackage{pdflscape}
\usepackage{booktabs}
\usepackage{dcolumn}
\usepackage{pdflscape}
\usepackage[hyphens]{url}
\usepackage{enumitem}
\usepackage[table]{xcolor}
\usepackage{authblk}
\usepackage{appendix}
\usepackage{titletoc}

\newcommand{\real}{\mathbb{R}_+^n}

\begin{document}

\section{The model}

Two states of the world: policies $x$ and $y$. Party $A$ and $B$ prefer $x$ and $y$ respectively. We can make this more explicit with
\begin{align*}
    U_A(x) > U_A(y)\\
    U_B(y) > U_B(x)
\end{align*}
For voters, $v_i > 0$ means voter $i$ prefers $x$ to $y$ All $i \in N$ vote for $x$ or $y$. Simple majority rule determines which policy gets implemented. For each voter, each party sets a bribe schedule
\begin{align*}
    a \in (a_1, ..., a_n) \in \mathbb{R}^n_+\\
    b \in (b_1, ..., b_n) \in \mathbb{R}^n_+
\end{align*}
Solving through backward induction, given bribe schedules $(a,b)$, voter $i$ prefers to vote for $x$ if $a_i + v_i > b_i$ and for $y$ otherwise. Since indifferent voters choose $y$, party $B$ needs to only match bribes from $A$, adjusting for individual voters' preferences: $b_i = a_i + v_i$. Therefore, $B$ solves

$$\min_C \left\{\sum_{i \in C} \max\{0, a_i + v_i\} : |C| > \frac{n}{2} \right\}$$

As long as this sum is strictly less than $W_B$; otherwise party $B$ chooses to set $b_i = 0 : \forall i \in N$.

Following Banks (2000), we restrict our analysis to the set of equilibria in which party $A$ wins, i.e. $W_A$ is sufficiently large relative to $\mathbf{v}$ and $W_B$ so that policy $x$ prevails over $y$. In other words, the folloing inequality must hold:

$$ \sum_{i \in C} \max{0, a_i + v_i} \geq W_B$$

Let $U(v, W_b) \subseteq \mathbb{R}^n_+$ denote the set of unbeatable bribe schedules. Additionally, let $S(a) = \sum^n_i a_i$ denote the bribe schedule for party $A$. The above assumptions on $W_A$, $W_B$ and $v$ guarantee that there is an

$$\tilde{a} \in U(\mathbf{v}, W_B) : S(\tilde{a}) \leq W_A$$

For party $A$, the solution is

\begin{equation}
\label{eqn:solution}
min\{S(a) : a \in U(\mathbf{v}, W_B) \}
\end{equation}

To fully describe the solution to equation \ref{eqn:solution}, we note the following: for any $a \in \mathbb{R}_+^n$, let $C(a) : {i \in N : a_i > 0}$ denote the set of individuals who receive a bribe from $A$. One can show that there is a bribe schedule $a'$ such that for any $i,j \in C(a)$, $a'_i + v_i = a'_j + v_j$. The intuition is that $A$ has no incentive to make voters differentially bribed, because $B$ will simply ignore the more expensive voters and target the weakest rings in the chain. Following Groseclose and Snyder (1996) we refer to this as a leveling schedule.

Let $U^l(\mathbf{v}, W_B) \subseteq U(\mathbf{v}, W_B)$ denote the set of unbeatable leveling schedules. These are bribe schedules such that $a_i + v_i = a_j + v_j  \equiv t(a)$. The bribe $a_i = t(a) - v_i$ is the sum of two terms. The first is the common "transfer" among all voters in $C(a)$, the second ($-v_i$) is individual specific. The latter term makes voters indifferent between $x$ and $y$ absent any bribe from $B$; the former represents the per capita amount necessary to make $C(a)$, together with any unbribed voters, unaffordable for $B$.

To further simplify the analysis, Banks introduces the folowing sets of assumption:

\begin{align*}
    A_1 &: v_{(n+1)/2} < 0\\
    A_2 &: v_1 < 2W_B/(n+1) 
\end{align*}

$A_1$ implies that absent any bribes by $A$, $y$ will defeat $x$. Therefore $A$ must bribe at least one voter. $A_2$ further implies that $A$ must bribe at least a majority of voters, otherwise $B$ will have sufficient resources to bribe $(n+1)/2$ voters and win.

Banks then proceeds to show that there are monotonic bribing schedules contained within the solution for equation \ref{eqn:solution}. For any $a \in \real$ let $k(a) = \lvert C(a) \rvert$. Suppose that $a\in U^l(\mathbf{v}, W_B)$ is such that $v_i \geq v_j$ and $j \in C(a)$ but $i \notin C(a)$. Then, under $A_2$, there exists $a' \in U^l(\mathbf{v}, W_B)$ with $S(a') \leq S(a)$, $k(a') = k(a)$ and $i \in C(a')$ but $j \notin C(a')$ by simply swapping $i$ for $j$. Note that since $v_i \geq v_j$, we have that $t(a) - v_i \leq  t(a) - v_j$, i.e. $a'_i \leq a_j$. $A_2$ guarantees that $a'_i \geq 0$, in other words every $i \in C(a)$ is receiving a non-negative bribe.

Generalizing, and recalling that $v_1 \geq ... \geq v_n$, we see that for all $a \in U^l(\mathbf{v}, W_b)$ there exists a bribe schedule $a' \in U^l(\mathbf{v}, W_b)$ such that $S(a') \leq S(a)$ and $C(a') = \{1, ..., k(a)\}$. Therefore, we can without loss of generality restrict attention to schedules $a$ by $A$ which bribe the first $k(a)$ voters. Call these monotonic leveling schedules and let $U_m^l \subseteq U(\mathbf{v}, W_B)$.

Therefore, when $A_2$ holds,

$$\min\{S(a): a \in U(\mathbf{v}, W_B)\} = \min\{S(a): a \in U^l_m(\mathbf{v}, W_B)\}$$

\end{document}